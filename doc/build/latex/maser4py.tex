% Generated by Sphinx.
\def\sphinxdocclass{report}
\documentclass[letterpaper,10pt,english]{sphinxmanual}
\usepackage[utf8]{inputenc}
\DeclareUnicodeCharacter{00A0}{\nobreakspace}
\usepackage{cmap}
\usepackage[T1]{fontenc}
\usepackage{babel}
\usepackage{times}
\usepackage[Bjarne]{fncychap}
\usepackage{longtable}
\usepackage{sphinx}
\usepackage{multirow}


\addto\captionsenglish{\renewcommand{\figurename}{Fig. }}
\addto\captionsenglish{\renewcommand{\tablename}{Table }}
\SetupFloatingEnvironment{literal-block}{name=Listing }



\title{maser4py Documentation}
\date{August 22, 2016}
\release{0.2.6}
\author{X.Bonnin, B.Cecconi, Q.N.Nguyen}
\newcommand{\sphinxlogo}{}
\renewcommand{\releasename}{Release}
\setcounter{tocdepth}{1}
\makeindex

\makeatletter
\def\PYG@reset{\let\PYG@it=\relax \let\PYG@bf=\relax%
    \let\PYG@ul=\relax \let\PYG@tc=\relax%
    \let\PYG@bc=\relax \let\PYG@ff=\relax}
\def\PYG@tok#1{\csname PYG@tok@#1\endcsname}
\def\PYG@toks#1+{\ifx\relax#1\empty\else%
    \PYG@tok{#1}\expandafter\PYG@toks\fi}
\def\PYG@do#1{\PYG@bc{\PYG@tc{\PYG@ul{%
    \PYG@it{\PYG@bf{\PYG@ff{#1}}}}}}}
\def\PYG#1#2{\PYG@reset\PYG@toks#1+\relax+\PYG@do{#2}}

\expandafter\def\csname PYG@tok@na\endcsname{\def\PYG@tc##1{\textcolor[rgb]{0.25,0.44,0.63}{##1}}}
\expandafter\def\csname PYG@tok@ne\endcsname{\def\PYG@tc##1{\textcolor[rgb]{0.00,0.44,0.13}{##1}}}
\expandafter\def\csname PYG@tok@ow\endcsname{\let\PYG@bf=\textbf\def\PYG@tc##1{\textcolor[rgb]{0.00,0.44,0.13}{##1}}}
\expandafter\def\csname PYG@tok@mo\endcsname{\def\PYG@tc##1{\textcolor[rgb]{0.13,0.50,0.31}{##1}}}
\expandafter\def\csname PYG@tok@bp\endcsname{\def\PYG@tc##1{\textcolor[rgb]{0.00,0.44,0.13}{##1}}}
\expandafter\def\csname PYG@tok@nn\endcsname{\let\PYG@bf=\textbf\def\PYG@tc##1{\textcolor[rgb]{0.05,0.52,0.71}{##1}}}
\expandafter\def\csname PYG@tok@sc\endcsname{\def\PYG@tc##1{\textcolor[rgb]{0.25,0.44,0.63}{##1}}}
\expandafter\def\csname PYG@tok@no\endcsname{\def\PYG@tc##1{\textcolor[rgb]{0.38,0.68,0.84}{##1}}}
\expandafter\def\csname PYG@tok@nl\endcsname{\let\PYG@bf=\textbf\def\PYG@tc##1{\textcolor[rgb]{0.00,0.13,0.44}{##1}}}
\expandafter\def\csname PYG@tok@s1\endcsname{\def\PYG@tc##1{\textcolor[rgb]{0.25,0.44,0.63}{##1}}}
\expandafter\def\csname PYG@tok@mh\endcsname{\def\PYG@tc##1{\textcolor[rgb]{0.13,0.50,0.31}{##1}}}
\expandafter\def\csname PYG@tok@nf\endcsname{\def\PYG@tc##1{\textcolor[rgb]{0.02,0.16,0.49}{##1}}}
\expandafter\def\csname PYG@tok@mf\endcsname{\def\PYG@tc##1{\textcolor[rgb]{0.13,0.50,0.31}{##1}}}
\expandafter\def\csname PYG@tok@gh\endcsname{\let\PYG@bf=\textbf\def\PYG@tc##1{\textcolor[rgb]{0.00,0.00,0.50}{##1}}}
\expandafter\def\csname PYG@tok@kt\endcsname{\def\PYG@tc##1{\textcolor[rgb]{0.56,0.13,0.00}{##1}}}
\expandafter\def\csname PYG@tok@err\endcsname{\def\PYG@bc##1{\setlength{\fboxsep}{0pt}\fcolorbox[rgb]{1.00,0.00,0.00}{1,1,1}{\strut ##1}}}
\expandafter\def\csname PYG@tok@ni\endcsname{\let\PYG@bf=\textbf\def\PYG@tc##1{\textcolor[rgb]{0.84,0.33,0.22}{##1}}}
\expandafter\def\csname PYG@tok@cpf\endcsname{\let\PYG@it=\textit\def\PYG@tc##1{\textcolor[rgb]{0.25,0.50,0.56}{##1}}}
\expandafter\def\csname PYG@tok@vg\endcsname{\def\PYG@tc##1{\textcolor[rgb]{0.73,0.38,0.84}{##1}}}
\expandafter\def\csname PYG@tok@gd\endcsname{\def\PYG@tc##1{\textcolor[rgb]{0.63,0.00,0.00}{##1}}}
\expandafter\def\csname PYG@tok@gs\endcsname{\let\PYG@bf=\textbf}
\expandafter\def\csname PYG@tok@nv\endcsname{\def\PYG@tc##1{\textcolor[rgb]{0.73,0.38,0.84}{##1}}}
\expandafter\def\csname PYG@tok@mb\endcsname{\def\PYG@tc##1{\textcolor[rgb]{0.13,0.50,0.31}{##1}}}
\expandafter\def\csname PYG@tok@ss\endcsname{\def\PYG@tc##1{\textcolor[rgb]{0.32,0.47,0.09}{##1}}}
\expandafter\def\csname PYG@tok@c1\endcsname{\let\PYG@it=\textit\def\PYG@tc##1{\textcolor[rgb]{0.25,0.50,0.56}{##1}}}
\expandafter\def\csname PYG@tok@ge\endcsname{\let\PYG@it=\textit}
\expandafter\def\csname PYG@tok@cp\endcsname{\def\PYG@tc##1{\textcolor[rgb]{0.00,0.44,0.13}{##1}}}
\expandafter\def\csname PYG@tok@gr\endcsname{\def\PYG@tc##1{\textcolor[rgb]{1.00,0.00,0.00}{##1}}}
\expandafter\def\csname PYG@tok@m\endcsname{\def\PYG@tc##1{\textcolor[rgb]{0.13,0.50,0.31}{##1}}}
\expandafter\def\csname PYG@tok@kd\endcsname{\let\PYG@bf=\textbf\def\PYG@tc##1{\textcolor[rgb]{0.00,0.44,0.13}{##1}}}
\expandafter\def\csname PYG@tok@vi\endcsname{\def\PYG@tc##1{\textcolor[rgb]{0.73,0.38,0.84}{##1}}}
\expandafter\def\csname PYG@tok@sx\endcsname{\def\PYG@tc##1{\textcolor[rgb]{0.78,0.36,0.04}{##1}}}
\expandafter\def\csname PYG@tok@sh\endcsname{\def\PYG@tc##1{\textcolor[rgb]{0.25,0.44,0.63}{##1}}}
\expandafter\def\csname PYG@tok@gp\endcsname{\let\PYG@bf=\textbf\def\PYG@tc##1{\textcolor[rgb]{0.78,0.36,0.04}{##1}}}
\expandafter\def\csname PYG@tok@il\endcsname{\def\PYG@tc##1{\textcolor[rgb]{0.13,0.50,0.31}{##1}}}
\expandafter\def\csname PYG@tok@nb\endcsname{\def\PYG@tc##1{\textcolor[rgb]{0.00,0.44,0.13}{##1}}}
\expandafter\def\csname PYG@tok@cm\endcsname{\let\PYG@it=\textit\def\PYG@tc##1{\textcolor[rgb]{0.25,0.50,0.56}{##1}}}
\expandafter\def\csname PYG@tok@mi\endcsname{\def\PYG@tc##1{\textcolor[rgb]{0.13,0.50,0.31}{##1}}}
\expandafter\def\csname PYG@tok@sr\endcsname{\def\PYG@tc##1{\textcolor[rgb]{0.14,0.33,0.53}{##1}}}
\expandafter\def\csname PYG@tok@cs\endcsname{\def\PYG@tc##1{\textcolor[rgb]{0.25,0.50,0.56}{##1}}\def\PYG@bc##1{\setlength{\fboxsep}{0pt}\colorbox[rgb]{1.00,0.94,0.94}{\strut ##1}}}
\expandafter\def\csname PYG@tok@vc\endcsname{\def\PYG@tc##1{\textcolor[rgb]{0.73,0.38,0.84}{##1}}}
\expandafter\def\csname PYG@tok@kp\endcsname{\def\PYG@tc##1{\textcolor[rgb]{0.00,0.44,0.13}{##1}}}
\expandafter\def\csname PYG@tok@o\endcsname{\def\PYG@tc##1{\textcolor[rgb]{0.40,0.40,0.40}{##1}}}
\expandafter\def\csname PYG@tok@gi\endcsname{\def\PYG@tc##1{\textcolor[rgb]{0.00,0.63,0.00}{##1}}}
\expandafter\def\csname PYG@tok@nd\endcsname{\let\PYG@bf=\textbf\def\PYG@tc##1{\textcolor[rgb]{0.33,0.33,0.33}{##1}}}
\expandafter\def\csname PYG@tok@ch\endcsname{\let\PYG@it=\textit\def\PYG@tc##1{\textcolor[rgb]{0.25,0.50,0.56}{##1}}}
\expandafter\def\csname PYG@tok@kc\endcsname{\let\PYG@bf=\textbf\def\PYG@tc##1{\textcolor[rgb]{0.00,0.44,0.13}{##1}}}
\expandafter\def\csname PYG@tok@gu\endcsname{\let\PYG@bf=\textbf\def\PYG@tc##1{\textcolor[rgb]{0.50,0.00,0.50}{##1}}}
\expandafter\def\csname PYG@tok@sd\endcsname{\let\PYG@it=\textit\def\PYG@tc##1{\textcolor[rgb]{0.25,0.44,0.63}{##1}}}
\expandafter\def\csname PYG@tok@kn\endcsname{\let\PYG@bf=\textbf\def\PYG@tc##1{\textcolor[rgb]{0.00,0.44,0.13}{##1}}}
\expandafter\def\csname PYG@tok@w\endcsname{\def\PYG@tc##1{\textcolor[rgb]{0.73,0.73,0.73}{##1}}}
\expandafter\def\csname PYG@tok@nc\endcsname{\let\PYG@bf=\textbf\def\PYG@tc##1{\textcolor[rgb]{0.05,0.52,0.71}{##1}}}
\expandafter\def\csname PYG@tok@s\endcsname{\def\PYG@tc##1{\textcolor[rgb]{0.25,0.44,0.63}{##1}}}
\expandafter\def\csname PYG@tok@nt\endcsname{\let\PYG@bf=\textbf\def\PYG@tc##1{\textcolor[rgb]{0.02,0.16,0.45}{##1}}}
\expandafter\def\csname PYG@tok@kr\endcsname{\let\PYG@bf=\textbf\def\PYG@tc##1{\textcolor[rgb]{0.00,0.44,0.13}{##1}}}
\expandafter\def\csname PYG@tok@si\endcsname{\let\PYG@it=\textit\def\PYG@tc##1{\textcolor[rgb]{0.44,0.63,0.82}{##1}}}
\expandafter\def\csname PYG@tok@s2\endcsname{\def\PYG@tc##1{\textcolor[rgb]{0.25,0.44,0.63}{##1}}}
\expandafter\def\csname PYG@tok@gt\endcsname{\def\PYG@tc##1{\textcolor[rgb]{0.00,0.27,0.87}{##1}}}
\expandafter\def\csname PYG@tok@go\endcsname{\def\PYG@tc##1{\textcolor[rgb]{0.20,0.20,0.20}{##1}}}
\expandafter\def\csname PYG@tok@k\endcsname{\let\PYG@bf=\textbf\def\PYG@tc##1{\textcolor[rgb]{0.00,0.44,0.13}{##1}}}
\expandafter\def\csname PYG@tok@sb\endcsname{\def\PYG@tc##1{\textcolor[rgb]{0.25,0.44,0.63}{##1}}}
\expandafter\def\csname PYG@tok@se\endcsname{\let\PYG@bf=\textbf\def\PYG@tc##1{\textcolor[rgb]{0.25,0.44,0.63}{##1}}}
\expandafter\def\csname PYG@tok@c\endcsname{\let\PYG@it=\textit\def\PYG@tc##1{\textcolor[rgb]{0.25,0.50,0.56}{##1}}}

\def\PYGZbs{\char`\\}
\def\PYGZus{\char`\_}
\def\PYGZob{\char`\{}
\def\PYGZcb{\char`\}}
\def\PYGZca{\char`\^}
\def\PYGZam{\char`\&}
\def\PYGZlt{\char`\<}
\def\PYGZgt{\char`\>}
\def\PYGZsh{\char`\#}
\def\PYGZpc{\char`\%}
\def\PYGZdl{\char`\$}
\def\PYGZhy{\char`\-}
\def\PYGZsq{\char`\'}
\def\PYGZdq{\char`\"}
\def\PYGZti{\char`\~}
% for compatibility with earlier versions
\def\PYGZat{@}
\def\PYGZlb{[}
\def\PYGZrb{]}
\makeatother

\renewcommand\PYGZsq{\textquotesingle}

\begin{document}

\maketitle
\tableofcontents
\phantomsection\label{index::doc}



\chapter{Introduction}
\label{intro:welcome-to-maser4py-s-documentation}\label{intro::doc}\label{intro:introduction}
The MASER python package (MASER4PY) contains modules to
deal with services and data provided in the framework
of the MASER portal.

For more information about MASER, please visit: \href{http://maser.lesia.obspm.fr/}{http://maser.lesia.obspm.fr/}


\chapter{Installation}
\label{intro:installation}

\section{System Requirements}
\label{intro:system-requirements}
In order to install MASER, make sure to have Python 3.4 or higher
available on your system.

The package installation requires the following Python modules:
- setuptools (12.0.5 or higher)
- openpyxl
- numpy (1.11.0 or higher)
- simplejson

MASER4PY has been tested on the following Operating Systems:
- Mac OS X 10.10, 10.11
- Debian Jessie 8.2

In order to use the ``cdf'' submodule, the NASA CDF software
distribution shall be installed and configured on your system.
Especially, make sure that the directory containing the CDF binary
executables is on your \$PATH, and the \$CDF\_LIB env. var. is set.


\section{How to get MASER4PY}
\label{intro:how-to-get-maser4py}
To download MASER-PY, enter the following command from a terminal:
\begin{quote}

git clone \href{https://git.obspm.fr/gitanonymous/projets/Plasma/maser4py}{https://git.obspm.fr/gitanonymous/projets/Plasma/maser4py}
\end{quote}

Make sure to have Git (\href{https://git-scm.com/}{https://git-scm.com/}) installed on your system.

If everything goes right, you should have a new local ``maser4py'' directory created on your disk.


\section{How to set up MASER4PY}
\label{intro:how-to-set-up-maser4py}
To set up the package on your system, enter the following
command from the ``maser4py'' directory:
\begin{quote}

python3 setup.py install
\end{quote}

This should install the maser4py package on your
system.

To check that the installation ends correctly, you can enter:
\begin{quote}

maser4py
\end{quote}

, which should return something like:
\begin{quote}

``This is maser4py package VX.Y.Z''
\end{quote}

If you have an issue durint installation, please read the ``Troubleshooting'' section for help.


\section{How to run MASER4PY}
\label{intro:how-to-run-maser4py}
If the installation has ended correctly, you can run MASER4PY:
\begin{itemize}
\item {} 
From a Python interpreter session, by entering ``import maser''.

\item {} 
Using the command line interface available for some MASER4PY modules.

\end{itemize}

For more details about the MASER4PY modules, please read the user manual.


\chapter{Overview}
\label{intro:overview}
The MASER4PY package is organized as follows:
\begin{quote}
\begin{description}
\item[{maser/}] \leavevmode\begin{description}
\item[{services/}] \leavevmode\begin{description}
\item[{helio/}] \leavevmode
Module to get and plot the HELIO Virtual Observatory data.

\end{description}

\item[{data/}] \leavevmode\begin{description}
\item[{wind/}] \leavevmode
Module to handle the Wind NASA mission data.

\end{description}

\item[{utils/}] \leavevmode\begin{description}
\item[{cdf/}] \leavevmode
Module to handle the NASA Common Data Format (CDF).

\item[{toolbox/}] \leavevmode
Module containing common tool methods for MASER-PY

\end{description}

\end{description}

\end{description}
\end{quote}

In order to work, the MASER4PY package modules rely on additional files and directories:
\begin{description}
\item[{maser/support}] \leavevmode
Directory containing support data

\end{description}


\chapter{The \emph{cdf} module}
\label{cdf::doc}\label{cdf:the-cdf-module}
The \emph{cdf} module is divided in two submodules:
\begin{itemize}
\item {} 
\emph{cdfconverter}, which allows users to convert CDF skeleton files into master CDF binary files

\item {} 
\emph{cdfvalidator}, which allows users to perform some validations on CDF files.

\end{itemize}

For more information about the CDF format, please visit \href{http://cdf.gsfc.nasa.gov/}{http://cdf.gsfc.nasa.gov/}.


\section{The \emph{cdfconverter} submodule}
\label{cdf:the-cdfconverter-submodule}
\emph{cdfconverter} contains the following classes:
\begin{itemize}
\item {} 
\emph{Xlsx2skt}, convert an Excel 2007 format file into a CDF skeleton table in the ASCII format. The organization of the Excel file shall follow some rules defined in the present document (see the section ``Excel file format definition'' below)

\item {} 
\emph{Skt2cdf}, convert a CDF skeleton table in ASCII format into a CDF master binary file. This module calls the ``skeletoncdf'' program from the NASA CDF software distribution.

\end{itemize}

Both classes can be imported from Python or called directly from a terminal using the dedicated command line interface.


\subsection{The Xlsx2skt class}
\label{cdf:the-xlsx2skt-class}
To import the Xlsx2skt class from Python, enter:

\begin{Verbatim}[commandchars=\\\{\}]
\PYG{k+kn}{from} \PYG{n+nn}{maser.utils.cdf.cdfconverter} \PYG{k+kn}{import} \PYG{n}{Xlsx2skt}
\end{Verbatim}


\subsubsection{Excel file format definition}
\label{cdf:excel-file-format-definition}
This section describes the organization of the input skeleton file in Excel format.

Note that:
\begin{itemize}
\item {} 
xlsx2skt supports the Excel 2007 format only (i.e., .xlsx).

\item {} 
Only zVariables are supported

\end{itemize}

\textbf{Make sure to respect the letter case, since the xlsx2skt parser is case sensitive!}

The input Excel file shall contain the following sheets:
\begin{itemize}
\item {} 
header

\item {} 
GLOBALattributes

\item {} 
zVariables

\item {} 
VARIABLEattributes

\item {} 
Options

\item {} 
NRV

\end{itemize}

The first row of each sheet shall be used to provide the name of the columns.


\paragraph{\emph{header} sheet}
\label{cdf:header-sheet}
The ``header'' sheet shall contain the following columns:
\begin{description}
\item[{CDF\_NAME}] \leavevmode
Name of the CDF master file (without the extension)

\item[{DATA ENCODING}] \leavevmode
Type of data encoding

\item[{MAJORITY}] \leavevmode
Majority of the CDF data parsing (``COLUMN'' or ``ROW'')

\item[{FORMAT}] \leavevmode
Indicates if the data are saved in a single (``SINGLE'') or
on multiple (``MULTIPLE'') CDF files

\end{description}


\paragraph{\emph{GLOBALattributes} sheet}
\label{cdf:globalattributes-sheet}
The ``GLOBALattributes'' sheet shall contain the following columns:
\begin{description}
\item[{Attribute Name}] \leavevmode
Name of the global attribute

\item[{Entry Number}] \leavevmode
Index of the current entry starting at 1

\item[{Data Type}] \leavevmode
CDF data type of the global attribute (only the ``CDF\_CHAR'' type is supported)

\item[{Value}] \leavevmode
Value of the current entry

\end{description}


\paragraph{\emph{zVariables} sheet}
\label{cdf:zvariables-sheet}
The ``zVariables'' sheet shall contain the following columns:
\begin{description}
\item[{Variable Name}] \leavevmode
Name of the zVariable

\item[{Data Type}] \leavevmode
CDF data type of the zVariable

\item[{Number Elements}] \leavevmode
Number of elements of the zVariable (shall be always 1, except for CDF\_{[}U{]}CHAR'' type)

\item[{Dims}] \leavevmode
Number of dimension of the zVariable (shall be 0 if the variable is a scalar)

\item[{Sizes}] \leavevmode
If the variable is not a scalar, provides its dimension sizes.

\item[{Record Variance}] \leavevmode
Indicates if the variable values can change (``T'') or not (``F'') from a record to another.

\item[{Dimension Variances}] \leavevmode
Indicates how the variable values vary over each dimension.

\end{description}


\paragraph{\emph{VARIABLEattributes} sheet}
\label{cdf:variableattributes-sheet}
The ``VARIABLEattributes'' sheet shall contain the following columns:
\begin{description}
\item[{Variable Name}] \leavevmode
Name of the zVariable

\item[{Attribute Name}] \leavevmode
Name of the variable attribute

\item[{Data Type}] \leavevmode
CDF data type of the variable attribute

\item[{Value}] \leavevmode
Value of the variable attribute

\end{description}


\paragraph{\emph{Options} sheet}
\label{cdf:options-sheet}
The ``Options'' sheet shall contain the following columns:
\begin{description}
\item[{CDF\_COMPRESSION}] \leavevmode
Type of compression of the CDF file (``None'' or empty field indicates no compression)

\item[{CDF\_CHECKSUM}] \leavevmode
Checksum algorithm of the CDF file (``None'' or empty field indicates no checksumming)

\item[{VAR\_COMPRESSION}] \leavevmode
Type of compression of each CDF variable (``None'' or empty field indicates no compression)

\item[{VAR\_SPARSERECORDS}] \leavevmode
value of sparese records (``None'' or empty field indicates no sparese value)

\item[{VAR\_PADVALUE}] \leavevmode
padvalue to provide to each variable. This option only works in the
case where all of the CDF variables has the same data type.
In the other cases, users should use the --Auto\_pad input keyword.

\end{description}


\paragraph{\emph{NRV} sheet}
\label{cdf:nrv-sheet}
The ``NRV'' sheet shall contain the following columns:
\begin{description}
\item[{Variable Name}] \leavevmode
Name of the zVariable

\item[{Index}] \leavevmode
Index of the current NR row

\item[{Value}] \leavevmode
Value of the current NR row

\end{description}


\subsubsection{Command line interface}
\label{cdf:command-line-interface}
To display the help of the module, enter:

\begin{Verbatim}[commandchars=\\\{\}]
\PYG{n}{xlsx2skt} \PYG{o}{\PYGZhy{}}\PYG{o}{\PYGZhy{}}\PYG{n}{help}
\end{Verbatim}

The full calling sequence is:

\begin{Verbatim}[commandchars=\\\{\}]
xlsx2skt [\PYGZhy{}h] [\PYGZhy{}O] [\PYGZhy{}V] [\PYGZhy{}Q] [\PYGZhy{}A] [\PYGZhy{}I] [\PYGZhy{}s [skeleton]] xlsx\PYGZus{}file
\end{Verbatim}

Input keyword list:
\begin{optionlist}{3cm}
\item [-h, -help]  
Display the module help
\item [-s, -{-}skeleton]  
skeleton
Name of the output skeleton table in ASCII format.
If not provided, use the name of the input file replacing the extension by `.skt'.
\item [-o, -{-}output\_dir]  
Path of the output directory. If not provided, use the directory of the input file.
\item [-A, -{-}Auto\_pad]  
If provided, the module will automatically set the pad values
(i.e, !VAR\_PADVALUE) for each CDF variable
\item [-I, -{-}Ignore\_none]  
If provided, the module will skip rows
for which the Attribute/Variable name columns are empty.
By default, the module returns an error if a empty Attribute/Variable name value is encountered.
\item [-O, -{-}Overwrite]  
Overwrite existing output ASCII skeleton table
\item [-V, -{-}Verbose]  
Talkative mode
\end{optionlist}


\subsubsection{Example}
\label{cdf:example}
To test the cdfconverter program, use the dedicated scripts/test\_cdfconverter.sh bash script.


\subsubsection{Limitations \& Known Issues}
\label{cdf:limitations-known-issues}
Here are some identified limitations to the module uses:
\begin{itemize}
\item {} 
Values provided in the ``Options'' sheet is valid for all of CDF file and variables. The module does not allow to set (yet) the values for each variable individually. \textbf{THUS, WE STRONGLY RECOMMEND TO USE THE --Auto\_pad INPUT KEYWORD (then edit the resulting skeleton table to modify the !VAR\_PADVALUE if required).}

\end{itemize}


\subsection{The \emph{Skt2cdf} class}
\label{cdf:the-skt2cdf-class}
To import the Skt2cdf class from Python, enter:

\begin{Verbatim}[commandchars=\\\{\}]
\PYG{k+kn}{from} \PYG{n+nn}{maser.utils.cdf.cdfconverter} \PYG{k+kn}{import} \PYG{n}{Skt2cdf}
\end{Verbatim}


\subsubsection{Command line interface}
\label{cdf:id1}
To display the help of the module, enter:

\begin{Verbatim}[commandchars=\\\{\}]
\PYG{n}{skt2cdf} \PYG{o}{\PYGZhy{}}\PYG{o}{\PYGZhy{}}\PYG{n}{help}
\end{Verbatim}

The full calling sequence is:

\begin{Verbatim}[commandchars=\\\{\}]
skt2cdf [\PYGZhy{}h] [\PYGZhy{}O] [\PYGZhy{}V] [\PYGZhy{}Q] [\PYGZhy{}s [executable]] [\PYGZhy{}c [output\PYGZus{}cdf]] skeleton
\end{Verbatim}

Input keyword list:
\begin{quote}
\begin{optionlist}{3cm}
\item [-h, -help]  
Display the module help
\item [-c, -{-}cdf]  
output\_cdf Name of the output CDF master binary file.
If not provided, use the name of the input file replacing the extension by `.cdf'.
\item [-o, -{-}output\_dir]  
Path of the output directory. If not provided, use the directory of the input file.
\item [-s, -{-}skeletoncdf executable]  
Path of the NASA GSFC CDF ``skeletoncdf'' executable.
If not provided, the program will search for the
executable in the \$PATH env. variable.
\item [-O, -{-}Overwrite]  
Overwrite existing output ASCII skeleton table
\item [-V, -{-}Verbose]  
Talkative mode
\item [-Q, -{-}Quiet]  
Quiet mode
\end{optionlist}
\end{quote}


\subsubsection{Example}
\label{cdf:id2}
To test the cdfconverter program, use the dedicated scripts/test\_cdfconverter.sh bash script.


\section{The \emph{cdfvalidator} submodule}
\label{cdf:the-cdfvalidator-submodule}
The \emph{cdfvalidator} submodule provides tools to validate a CDF format file.

It contains only one \emph{Validate} class that regroups all of the validation methods.


\subsection{The \emph{Validate} class}
\label{cdf:the-validate-class}
To import the \emph{Validate} class from Python, enter:

\begin{Verbatim}[commandchars=\\\{\}]
\PYG{k+kn}{from} \PYG{n+nn}{maser.utils.cdf.cdfvalidator} \PYG{k+kn}{import} \PYG{n}{Validate}
\end{Verbatim}


\subsubsection{The Model validation test}
\label{cdf:the-model-validation-test}
The \emph{Validate} class allows user to check if a given CDF format file contains specific attributes or variables, by providing a
so-called ``cdfvalidator model file''.

This model file shall be in the JSON format. All items and values are case sensitive.
It can include the following JSON objects:


\begin{threeparttable}
\capstart\caption{CDFValidator JSON objects}\label{cdf:id5}
\begin{tabulary}{\linewidth}{|L|L|}
\hline
\textsf{\relax 
JSON object
} & \textsf{\relax 
Description
}\\
\hline
GLOBALattributes
 & 
Contains the list of global attributes to check
\\
\hline
VARIABLEattributes
 & 
Contains the list of variable attributes to check
\\
\hline
zVariables
 & 
Contains the list of zvariables to check
\\
\hline\end{tabulary}

\end{threeparttable}


Note that any additional JSON object will be ignored.

The table below lists the JSON items that are allowed to be found in the \emph{GLOBALattributes}, \emph{VARIABLEattributes} and \emph{zVariables} JSON objects.


\begin{threeparttable}
\capstart\caption{CDFValidator JSON object items}\label{cdf:id6}
\begin{tabulary}{\linewidth}{|L|L|L|L|}
\hline
\textsf{\relax 
JSON item
} & \textsf{\relax 
JSON type
} & \textsf{\relax 
Priority
} & \textsf{\relax 
Description
}\\
\hline
attributes
 & 
vector
 & 
optional
 & 
List of variable attributes. An element of the vector shall be a JSON object that can contain one or more of the other  JSON items listed in this table
\\
\hline
dims
 & 
integer
 & 
optional
 & 
Number of dimensions of the CDF item
\\
\hline
entries
 & 
vector
 & 
optional
 & 
Entry value(s) of the CDF item to be found
\\
\hline
hasvalue
 & 
boolean
 & 
optional
 & 
If it is set to true, then the current CDF item must have at least one nonzero entry value
\\
\hline
name
 & 
string
 & 
mandatory
 & 
Name of the CDF item (attribute or variable) to check
\\
\hline
sizes
 & 
vector
 & 
optional
 & 
Dimension sizes of the CDF item
\\
\hline
type
 & 
attribute
 & 
optional
 & 
CDF data type of the CDF item
\\
\hline\end{tabulary}

\end{threeparttable}



\subsubsection{Command line interface}
\label{cdf:id3}
To display the help of the module, enter:

\begin{Verbatim}[commandchars=\\\{\}]
\PYG{n}{cdfvalidator} \PYG{o}{\PYGZhy{}}\PYG{o}{\PYGZhy{}}\PYG{n}{help}
\end{Verbatim}

The full calling sequence is:

\begin{Verbatim}[commandchars=\\\{\}]
cdfvalidator [\PYGZhy{}\PYGZhy{}help] [\PYGZhy{}\PYGZhy{}Verbose] [\PYGZhy{}\PYGZhy{}Quiet] [\PYGZhy{}\PYGZhy{}log\PYGZus{}file [log\PYGZus{}file]] \PYGZbs{}
[\PYGZhy{}\PYGZhy{}ISTP] [\PYGZhy{}\PYGZhy{}CDFValidate [executable]] [\PYGZhy{}\PYGZhy{}model\PYGZus{}file [model\PYGZus{}file]] skeleton
\end{Verbatim}

Input keyword list:
\begin{optionlist}{3cm}
\item [-h, -help]  
Display the module help
\item [-l, -{-}log\_file]  
Path of the output log file.
\item [-I, -{-}ISTP]  
Perform the ISTP compliance validation test
\item [-m, -{-}model\_file]  
Path to the input model file in JSON format
(see ``Model validation test'' section for more information).
\item [-C, -{-}CDFValidate executable]  
Path of the NASA GSFC CDF ``CDFValidate'' executable.
If it is not provided, the module will
search in the directories defined in \%\%\$PATH\%\%.
\item [-Q, -{-}Quiet]  
Quiet mode
\item [-V, -{-}Verbose]  
Talkative mode
\end{optionlist}


\subsubsection{Example}
\label{cdf:id4}
To test the cdfvalidator program, use the dedicated scripts/test\_cdfvalidator.sh bash script.

It should return something like:

\begin{Verbatim}[commandchars=\\\{\}]
INFO    : Opening /tmp/cdfconverter\PYGZus{}example.cdf
INFO    : Loading /Users/xbonnin/Work/projects/MASER/Software/Tools/Git/maser\PYGZhy{}py/scripts/../maser/support/cdf/cdfvalidator\PYGZus{}model\PYGZus{}example.json
INFO    : Checking GLOBALattributes:
INFO    : \PYGZhy{}\PYGZhy{}\PYGZgt{} Project
WARNING : \PYGZdq{}Project\PYGZdq{}  has a wrong entry value: \PYGZdq{}Python\PYGZgt{}Python 2\PYGZdq{} (\PYGZdq{}Python\PYGZgt{}Python 3\PYGZdq{} expected)!
INFO    : \PYGZhy{}\PYGZhy{}\PYGZgt{} PI\PYGZus{}name
INFO    : \PYGZhy{}\PYGZhy{}\PYGZgt{} TEXT
INFO    : Checking VARIABLEattributes:
INFO    : \PYGZhy{}\PYGZhy{}\PYGZgt{} FIELDNAM
INFO    : \PYGZhy{}\PYGZhy{}\PYGZgt{} CATDESC
INFO    : \PYGZhy{}\PYGZhy{}\PYGZgt{} VAR\PYGZus{}TYPE
INFO    : Checking zVariables:
INFO    : \PYGZhy{}\PYGZhy{}\PYGZgt{} Epoch
INFO    : \PYGZhy{}\PYGZhy{}\PYGZgt{} Variable2
INFO    : Checking variable attributes of \PYGZdq{}Variable2\PYGZdq{}:
INFO    : \PYGZhy{}\PYGZhy{}\PYGZgt{} DEPEND\PYGZus{}0
WARNING : DEPEND\PYGZus{}0 required!
INFO    : Closing /tmp/cdfconverter\PYGZus{}example.cdf
\end{Verbatim}


\chapter{The \emph{wind} module}
\label{wind::doc}\label{wind:the-wind-module}
The Wind module provides methods to deal with the Wind NASA mission data.


\section{The \emph{waves} submodule}
\label{wind:the-waves-submodule}

\chapter{The \emph{stereo} module}
\label{stereo::doc}\label{stereo:the-stereo-module}
The stereo module provides methods to deal with the STEREO NASA mission data.


\section{The \emph{swaves} submodule}
\label{stereo:the-swaves-submodule}

\chapter{The \emph{helio} module}
\label{helio::doc}\label{helio:the-helio-module}
The helio module provides methods to deal with the HELIO Virtual Observatory services and data.


\section{The \emph{hfc} submodule}
\label{helio:the-hfc-submodule}

\chapter{Troubleshooting}
\label{appendices:troubleshooting}\label{appendices::doc}
Here a list of known issues.
If the problem persists, you can contact the MASER developer team at: \href{mailto:maser.support@groupes.renater.fr}{maser.support@groupes.renater.fr}.
\begin{itemize}
\item {} 
//I have an error message ``{[}Errno 13{]} Permission denied:'' during installation//:

\end{itemize}

This means that you don't have the right to install the package in your Python ``site-packages'' local directory.
To solve this problem, install the package as a super user (e.g., using sudo command for instance), or modify the ``site-packages'' user access permissions.
\begin{itemize}
\item {} 
//I have an error message ``Exception: Cannot find CDF C library. Try os.putenv(``CDF\_LIB'', library\_directory) before import.'' during the installation''//:

\end{itemize}

This means that the \%\%\$CDF\_LIB\%\% environmment variable is not set. This variable is required to run the CDF software distribution from Python. For more information about how to set up this software., visit the CDF home page at \href{http://cdf.gsfc.nasa.gov/}{http://cdf.gsfc.nasa.gov/}.


\chapter{Indices and tables}
\label{index:indices-and-tables}\begin{itemize}
\item {} 
\DUspan{xref,std,std-ref}{genindex}

\item {} 
\DUspan{xref,std,std-ref}{modindex}

\item {} 
\DUspan{xref,std,std-ref}{search}

\end{itemize}



\renewcommand{\indexname}{Index}
\printindex
\end{document}
